\documentclass[12pt,a4paper]{article}
\usepackage[utf8]{inputenc}
\usepackage[spanish]{babel}
\usepackage{amsmath}
\usepackage{amsfonts}
\usepackage{amssymb}
\title{Tarea 2 Denavit Hartenberg}
\author{Amaury Efrain Gutierrez Chavez}
\begin{document}
\maketitle
Se trata de un procedimiento sistemático para describir la estructura cinemática de un brazo. El cual cuenta con varios pasos para realizarlo, los cuales son los siguientes

\begin{description}
\item[1- Numerar los eslabones:]Se enumeran los eslabones  de 0 a n, comenzando desde la base  como 0
\item[2- Numerar las articulaciones:]Igual que los eslabones se enumeran de 1 a n
\item[3- Localizar los ejes de las articulaciones:]Si es rotacional será el eje de giro, y si es prismática será el eje a lo largo del cual se produce el desplazamiento.
\item[4- Ejes Z:]Empezamos a colocar los sistemas XYZ. Situamos los Zi en los ejes de las articulaciones i, con i=1,…,n. Es decir, Z0 va sobre el eje de la 1ª articulación, Z1 va sobre el eje del 2º grado de libertad, etc.
\item[5- Sistema de coordenadas 0:]Se sitúa el punto origen en cualquier punto a lo largo de Z0. La orientación de X0 e Y0 puede ser arbitraria, siempre que se respete evidentemente que XYZ sea un sistema dextrógiro.
\item[6- Resto de sistemas:] Para el resto de sistemas i=1,…,N-1, colocar el punto origen en la intersección de Zi con la normal común a Zi y Zi+1. En caso de cortarse los dos ejes Z, colocarlo en ese punto de corte. En caso de ser paralelos, colocarlo en algún punto de la articulación i+1.
\item[7- Ejes X:]Cada Xi va en la dirección de la normal común a Zi-1 y Zi, en la dirección de Zi-1 hacia Zi.
\item[8- Ejes Y:]Una vez situados los ejes Z y X, los Y tienen su dirección determianadas por la restricción de formar un XYZ dextrógiro.
\item[9- Sistema del extremo del robot:]El n-ésimo sistema XYZ se coloca en el extremo del robot (herramienta), con su eje Z paralelo a Zn-1 y X e Y en cualquier dirección válida.
\item[10- Ángulos teta:]Cada 0i es el ángulo desde Xi-1 hasta Xi girando alrededor de Zi.
\item[11-	Distancias d:]Cada  di es la distancia desde el sistema XYZ i-1 hasta la intersección de las normales común de  Zi-1 hacia Zi, a lo largo de  Zi-1.
\item[12-	Distancias a:]Cada  ai es la longitud de dicha normal común.
\item[13-	Ángulos alfa:]Ángulo que hay que rotar Zi-1 para llegar a Zi, rotando alrededor de Xi.
\item[14-	Matrices individuales:]Cada eslabón define una matriz de transformación 
\item[15-	Transformación total:]La matriz de transformación total que relaciona la base del robot con su herramienta es la encadenación (multiplicación) de todas esas matrices
\end{description}

\end{document}