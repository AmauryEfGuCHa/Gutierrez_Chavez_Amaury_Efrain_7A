\documentclass[12pt,a4paper]{article}
\usepackage[utf8]{inputenc}
\usepackage[spanish]{babel}
\usepackage{amsmath}
\usepackage{amsfonts}
\usepackage{amssymb}
\title{Tarea: Aplicaciones de los manipuladores  }
\author{Amaury Efrain Gutierrez Chavez}
\begin{document}
\maketitle
En el cuerpo humano una cadena cinematica abierta (robot serial) puede ser vista como un solo brazo con el cual se puede realizar diversas tareas. Sin embargo, si se requiere de mayor fuerza o de precisión el ser humano emplea los dos brazos. Al utilizar dos brazos para manipular una pieza esto seria el concepto de robot paralelo, asi se gana mayor capacidad de carga y precisión pero a expensas de perder espacio de trabajo.\\
Bonev establece que el origen del robot paralelo se encuentra en la industria del entretenimiento, siendo James E. Gwinnett en el año 1928 uno de los pioneros en patentar un artefacto basado en el concepto de robot paralelo. El dispositivo 
presenta una arquitectura donde una cadena cinemática o pierna central restringe el movimiento de la plataforma móvil respecto a la base de forma tal que su movimiento resultante es del tipo esférico. Una cadena cinemática ubicada en uno de los extremos de las plataformas, provee el movimiento de rotación a la plataforma móvil que es aprovechado para producir el movimiento requerido para el entretenimiento de los usuarios.\\

Por parte de la industria el primero fue presentada por Willard L.V. Pollard en el año 1940. El dispositivo fue propuesto para pintar vehículos de forma automática con pintura de aerosol y posteriormente fue patentado como dispositivo para controlar el posicionamiento de una herramienta\\
Más tarde en la década de los 50, Eric Gough un ingeniero automotriz trabajando en la fábrica de neumáticos de la Ford Dunlop en Birmingham, Inglaterra, desarrolla una máquina Universal de pruebas de neumáticos . La plataforma fue puesta en funcionamiento en el año 1954 y cumplía la función de probar mecánicamente neumáticos mediante la aplicación de cargas combinadas.\\
Luego de la introducción de la configuración de robot paralelo para desarrollar simuladores de vuelo, la aplicación más conocida y desarrollada de estos robot es en operaciones de pick and place.\\

El primer robot comercial y posiblemente uno de los más exitosos en implementación industrial lo constituye el robot Delta desarrollado a partir de la década de los 80 en la École Polytechnique Fédérale de Lausanne (Suiza) por Reymond Clavel [12]. El robot fue desarrollado partiendo de la idea de desarrollar un robot para manipular objetos de bajo peso a altas velocidades. La particularidad del robot Delta es que la plataforma móvil va unida a la base mediante 3 piernas donde cada pierna presenta un mecanismo de paralelogramo que permite balancear el centro de masa de cada una de ellas.\\
Las principales aplicaciones del robot antes descritos se encuentran en Industrias de empaquetado como la de alimentos, para el manejo de células fotovoltaicas, manejo de instrumentos médicos, corte laser a alta velocidad, así como también mecanizado de piezas de madera. También han sido empleados para la manejo de verduras, para mayor detalle se refiere al lector a la referencia\\

Otra de las aplicaciones prácticas de los robots paralelos se encuentra en el área de desarrollo de centros de mecanizado. En este campo es común que los desarrolladores se refieran al robot paralelo como Mecanismo Cinemático Paralelo (MCP) en inglés Parallel Kinematic Mechanism (PKM). El primer prototipo de centro de mecanizado basado en la en MCP fue presentado al público en el evento International Manufacturing Technology Show (IMTS) de 1994 en Chicago, USA. El dispositivo utiliza un robot paralelo del tipo plataforma Stewart (6GdL) para realizar operaciones de mecanizado en 5 ejes. A partir su introducción en 1994 el número de desarrollo y patentes de centros de mecanizados basados en MCP ha incrementado constantemente.\\

La cirugía mínimamente invasiva representa una de las áreas donde la introducción de robot 
produce un gran impacto, sobre todo mejorando las prestaciones de la cirugía laparoscópica, ya que aumenta la habilidad del cirujano a la hora de realizar una operación (mayor precisión, evita el movimiento errático del pulso de la mano). Con la cirugía robótica se han logrado avances como realizar una operación mediante orificios de 10 mm en el cuerpo del paciente. En la actualidad solo hay un robot comercial disponible que es el sistema Da Vinci. El sistema se ha comercializado a partir de los años 90 y en el año 2000 fue autorizado por la Administración de Alimentos y Medicamentos (FDA) de los Estados Unidos. El robot consiste de 3 o 4 brazos robóticos del tipo serial, una consola o monitor para interacción del médico con el robot y una camilla donde se ubica al paciente


\end{document}